% CV in LaTeX
\documentclass[10pt]{article}

%% PACKAGES %%
\usepackage[english]{babel}
\usepackage[utf8]{inputenc}
\usepackage{amsmath,amssymb,amsfonts,amsthm}
\usepackage{geometry}
\usepackage{tgpagella}

\geometry{
	a4paper,
	left=15mm,
	right=15mm,
	top=10mm,
	bottom=10mm
}

%\geometry{
%	a4paper,
%	left=20mm,
%	right=20mm,
%	top=20mm,
%	bottom=20mm
%}

\pagestyle{empty}

\usepackage{titlesec}
\usepackage{xcolor}

% Section formatting
%\usepackage{sectsty}
%\sectionfont{%
%	\large
%	\fontfamily{qag}\selectfont
%	\sectionrule{0pt}{0pt}{-5pt}{1pt}
%}
% Section formatting with titlesec (replacing sectsty)

% Áp dụng định dạng tương tự cho \section* (không đánh số)
\titleformat{name=\section,numberless}
[block]
{\normalfont\Large\bfseries\fontfamily{qag}\selectfont}
{}  % không có số
{0em}
{}
[\titlerule]

\titlespacing*{\section}
{-0.5cm}    % ← lùi vào 1cm
{2ex plus 1ex minus .2ex}
{2.5ex plus 1ex minus .2ex}


\usepackage[shortlabels]{enumitem}
\newlist{soloitemize}{itemize}{1}
\setlist[soloitemize]{
	nosep,
	topsep=0pt,
	itemsep=4pt,
	left=2pt,
	label=$\bullet$,
}

% For positioning and boxes
\usepackage[absolute,overlay]{textpos}
\usepackage{tcolorbox}

% Misc packages
\usepackage{calc}
\usepackage[shortcuts]{extdash}
\usepackage{xcolor}
\usepackage{graphicx}
\usepackage{longtable}
\usepackage{hyperref}
\hypersetup{
	colorlinks=true,
	linkcolor=blue,
	urlcolor=blue,
	citecolor=blue,
	pdfborder={0 0 0}
}
\usepackage{soul}
\definecolor{softyellow}{RGB}{255, 255, 180} % very light yellow
\sethlcolor{softyellow}

\usepackage{tikz}

% Underlined hyperlinks
\let\oldhref\href
\renewcommand{\href}[2]{\oldhref{#1}{\ul{#2}}}

% Separator line command
\newcommand{\sepspace}{%
	\par\vspace{0.5em}
	\noindent
	\tikz{\draw[gray, dashed, line width=0.5pt] (0,0) -- (\linewidth,0);}
	\par\vspace{0.5em}
}

% Length for alignment boxes
\newlength{\spacebox}
\settowidth{\spacebox}{12345678}

%% MACROS %%

% Name
\newcommand{\name}[1]{%
	\Huge
	\fontfamily{phv}\selectfont
	\textbf{#1}%
	\par\normalsize\normalfont
}

% Personal info entries
\newcommand{\info}[2]{%
	\noindent\hangindent=1em\hangafter=0
	\parbox{\spacebox}{\textsl{#1}} #2 
}

% Skill entry


\newlength{\skilllabelwidth}
\settowidth{\skilllabelwidth}{\textbf{12345678912345123441}}

\newcommand{\skill}[2]{%
	\noindent
	\parbox[t]{\skilllabelwidth}{\raggedright\textbf{#1}}%
	\hspace{0.75em}% wider gap between label and text
	\parbox[t]{\dimexpr\linewidth-\skilllabelwidth-0.75em\relax}{%
		\setlength{\baselineskip}{1.35\baselineskip}% line spacing inside detail
		#2%
	}\par\vspace{0.7em} % space between skill entries
}

% Language level
\newcommand{\lan}[2]{%
	\noindent\hangindent=2em\hangafter=0
	\parbox{\spacebox}{\textbf{#1}} #2 \par
}

% Education entry
\newcommand{\education}[4]{%
	\noindent \textbf{#1}, \textit{#3} \hfill \textit{#2}\par
	%\vspace{0.5em}
	%\noindent \textit{\vspace{0.15cm}#3}\par
	\vspace{0.5em}
	\noindent\hangindent=2em\hangafter=0 #4 \par\normalsize
}

% Work experience entry
\newcommand{\work}[4]{%
	\noindent \textbf{#1} \hfill \textit{#2}\par
	\vspace{0.5em}
	\noindent \textit{\vspace{0.15cm}#3}\par
	\vspace{0.5em}
	\noindent\hangindent=2em\hangafter=0 #4 \par\normalsize
}

% Project entry
\newcommand{\project}[4]{%
	\noindent \textbf{#1} \hfill \textit{#2}\par
	\vspace{0.5em}
	\noindent \textit{\vspace{0.15cm}#3}\par
	\vspace{0.5em}
	\noindent\hangindent=2em\hangafter=0 #4 \par\normalsize
}

% Publication entry
\newcommand{\publication}[5]{%
	\noindent \textbf{#1} \hspace{0.1cm} #2 \par
	\vspace{0.5em}
	\noindent #3 [\href{https://doi.org/#5}{pdf}] \par
	\vspace{0.5em}
	\noindent \textit{#4}
}


% Conference entry
\newcommand{\conference}[2]{%
	\noindent \textbf{#1} \par
	\vspace{0.5em}
	\noindent #2 \par
}

\newcommand{\activities}[2]{%
	\noindent \textbf{#1} \par
	\vspace{0.5em}
	\noindent #2 \par
}

%\newcommand{\skill}[2]{%
%	\noindent \textbf{#1} \par
%	\vspace{0.3em}
%	\noindent\hangindent=2em\hangafter=0 #2 \par
%	\vspace{0.8em}
%}


\graphicspath{{../images/}}


\begin{document}
	
	\begin{textblock*}{5cm}(15.5cm,0.8cm) 
		\centering
		\begin{tcolorbox}[colframe=black, colback=white, sharp corners]
			\fontfamily{phv}\selectfont \centering\footnotesize \textit{Last update: Aug 2025} \normalsize\normalfont
		\end{tcolorbox}
	\end{textblock*}
	
	
	\name{Duc-Cuong VU, BSc.}
	\sepspace
	\info{Email}{\href{mailto:vdcuong2002@gmail.com}{\texttt{vdcuong2002 [at] gmail.com}}} \hspace{2cm}
	\info{Website}{\href{https://duccuongvu.github.io}{https://duccuongvu.github.io}}
	\vspace{0.2em}\par
	
	
	\section*{Education}
	
	\education{Master of Sci. in Automation and Control}{Jul 2024 - present}{%School of Electrical - Electronics Engineering,
		Hanoi University of Science and Technology (\href{https://hust.edu.vn/en/}{HUST}),
		%\hfill Hanoi, Vietnam
	}
	{
		\begin{soloitemize}
			\item \textbf{Research project:} \textit{Design control structures for Parallel Platforms in Maritime applications}
			\item \textbf{Funded by:} \textit{Master, PhD Scholarship Programme of Vingroup Innovation Foundation (\href{https://vinif.org/en/}{VinIF})} 
		\end{soloitemize}
	}
	
	\sepspace
	
	\education{Bachelor of Sci. in Automation and Control}
	{Oct 2020 - Mar 2024}{
		%School of Electrical - Electronics Engineering, 
		Hanoi University of Science and Technology (\href{https://hust.edu.vn/en/}{HUST}), 
		%\hfill Hanoi, Vietnam
	}
	{\begin{soloitemize}
			\item \textbf{Excellent degree}, GPA: 3.71/4 (Rank: 22/499). Finished the 4-year BSc program in \textbf{just 3.5 years}.
			\item \textbf{Thesis:} \textit{Balancing, motion planning, and tracking control for ballbot systems} [\href{https://drive.google.com/file/d/14nDBzQam5qdcvj9y6AuS6N0fQ292AwWj/view?usp=sharing}{pdf}] (The best thesis defense)
		\end{soloitemize}
	}
	
	

	
	\section*{Selected publications}
	
	\publication{Journal}
	{Ocean Engineering (SCIE Q1) (2025)}
	{Glocal trajectory generation and tracking control for AUVs with optimal coverage sensor networks}
	{\hl{\textbf{Duc Cuong Vu}}*, Son Tran*, Tung Lam Nguyen, and Duc Chinh Hoang}
	{10.1016/j.oceaneng.2025.122902}
	
	\sepspace
	
	\publication{Journal}
	{Ocean Engineering (SCIE Q1) (2025)}
	{Lagrangian-based modeling and safety-critical controls for Stewart platforms under marine operations}
	{\hl{\textbf{Duc Cuong Vu}}, Danh Huy Nguyen, Minh Nhat Vu, and Tung Lam Nguyen}
	{10.1016/j.oceaneng.2025.122142}
	
	\sepspace
	
	\publication{Journal}
	{IEEE Acess (SCIE Q2) (2025)}
	{CBFs-based Model Predictive Control for Obstacle Avoidance with Tilt Angle Limitation for Ball-Balancing Robots}
	{Minh Duc Pham, \hl{\textbf{Duc Cuong Vu}}, Thi Thuy Hang Nguyen, Thi Van Anh Nguyen, Minh Nhat Vu, and Tung Lam Nguyen}
	{10.1109/ACCESS.2025.3567474}
	
	\sepspace
	
	
	\publication{Journal}
	{Results in Engineering (ESCI Q1) (2025)}
	{A novel approach of Consensus-based Finite-time Distributed Sliding Mode Control for Stewart platform manipulators motion tracking}
	{\hl{\textbf{Duc Cuong Vu}}, Danh Huy Nguyen, and Tung Lam Nguyen}
	{10.1016/j.rineng.2024.103872}
	
	\sepspace
	\publication{Journal}
	{International Journal of Robust and Nonlinear Control (SCIE Q1) (2024)}
	{Time-optimal trajectory generation and observer-based hierarchical sliding mode control for ballbots with system constraints}
	{\hl{\textbf{Duc Cuong Vu}}, Minh Duc Pham, Thi Thuy Hang Nguyen, Thi Van Anh Nguyen, and Tung Lam Nguyen}
	{10.1002/rnc.7358}
	
	
		% work experience
	\section*{Work experience}
	
	\work{VinRobotics}
	{Sep 2025 - present}
	{Robotics Engineer \hfill Hanoi, Vietnam}
	{\begin{soloitemize}
			\item Responsible for System Identification, State Estimation, Model Predictive Control (MPC), and Whole Body Control (WBC) for VinRobotics Humanoids robot.
	\end{soloitemize}}
%	
	\sepspace
	
	\work{Mechatronics Engineering Group at HUST}
	{Oct 2021 - present}
	{Research Assistant supervised by \href{https://scholar.google.com/citations?user=MlJ_2-wAAAAJ&hl=en}{\textit{Assoc.Prof.PhD. Tung Lam Nguyen}},\hfill Hanoi, Vietnam}
	{ \begin{soloitemize}
			\item Conducted research on advanced control strategies, robotics, motion control, and multi-agent systems, focusing on both theoretical development and practical implementation.
		\end{soloitemize}
	}
	
		

	


	\section*{Academic projects}
		\project{Advanced Control of a Ship-Mounted Stewart Platform for Marine Applications}
		{Mar 2025 - Dec 2025}
		{Research assistant supervised by \href{https://scholar.google.com/citations?user=qyExc4QAAAAJ&hl=en}{\textit{PhD. Minh Nhat Vu}} (PI) and \href{https://scholar.google.com/citations?user=MlJ_2-wAAAAJ&hl=en}{\textit{Assoc.Prof.PhD. Tung Lam Nguyen}}}
		{\begin{soloitemize}
				\item Field: Marine Robotics and Control Systems.
				\item International Collaboration of Korea Institute of Science and Technology and Institute (\href{https://www.kist.re.kr/eng/index.do}{KIST}) for Control Engineering and Automation (\href{https://hust.edu.vn/en/}{HUST}) via the \textit{KIST School Partnership Project}.
%				\item Designing and implementing advanced control algorithms for the Stewart platform, including safety-critical and robust control strategies tailored for marine environments.
%				\item Developing high-fidelity simulation (Simscape, MuJoCo) that capture marine environmental disturbances (such as waves, currents, and ship motion) and accurately represent the platform's kinematics and dynamics.
%				\item Building the experimental setup, including mechanical assembly, hardware integration, Linux-based real-time kernel configuration, and EtherCAT communication for precise control and data acquisition.
%				\item Collaborating with cross-institutional teams to refine system requirements, troubleshoot technical challenges, and ensure seamless integration of hardware and software components.
%				\item Preparing detailed technical documentation, authoring scientific publications, including Ocean Engineering (OE) and Results in Engineering, and presenting project outcomes to both academic and industrial collaborators.
				\item Designed and implemented advanced control algorithms for a Stewart platform in marine environments, supported by high-fidelity simulations (Simscape, MuJoCo) and validated through a full experimental setup (mechanical assembly, hardware integration, Linux real-time kernel, EtherCAT communication).
				\item Collaborated with cross-institutional teams on system integration, troubleshooting, and documentation, while authoring peer-reviewed publications (Ocean Engineering, Results in Engineering) and presenting outcomes to academic and industrial partners.
			\end{soloitemize}
		}
	
		\sepspace
		
		\project{Robot navigation system integrating sensor network and wireless communication}{Jan 2025 - Dec 2027}{Research assistant supervised by \href{https://scholar.google.com/citations?user=mI561CkAAAAJ&hl=en}{\textit{PhD. Chinh Hoang Duc}} (PI)
			and \href{https://scholar.google.com/citations?user=MlJ_2-wAAAAJ&hl=en}{\textit{Assoc.Prof.PhD. Tung Lam Nguyen}}.}
		{\begin{soloitemize}
				\item Field: Communications, Optimization, Robotics, and Control Systems.
				\item Funded by Hanoi University of Science and Technology (\href{https://hust.edu.vn/en/}{HUST}).
%				\item Designing and developing a comprehensive simulation environment for Autonomous Underwater Vehicles (AUVs) using the MuJoCo physics engine, enabling accurate modeling of underwater dynamics, sensor feedback, and environmental disturbances.
%				\item Implementing and validating advanced control algorithms for robust navigation, obstacle avoidance, and trajectory tracking in challenging underwater scenarios.
%				\item Integrating sensor network data and wireless communication protocols into the simulation framework to evaluate system performance under realistic communication constraints.
%				\item Collaborating with team members to troubleshoot technical challenges, optimize simulation fidelity, and ensure seamless integration between sensing and control systems.
%				\item Documenting research findings, preparing technical reports, and authoring a peer-reviewed scientific paper for submission to an international journal or conference based on the project outcomes.
				\item Designed and developed a comprehensive MuJoCo-based simulation environment for AUVs, incorporating underwater dynamics, sensor feedback, environmental disturbances, and communication constraints to evaluate system performance.
				\item Implemented and validated advanced control algorithms for navigation, obstacle avoidance, and trajectory tracking, while collaborating on integration, troubleshooting, and authoring a peer-reviewed scientific paper.
			\end{soloitemize}
		}
	
		\sepspace
		
		\project{Balancing, motion planning, and tracking control for ballbot systems}{Jul 2023 - Jul 2024}{Bachelor graduated project supervised by \href{https://scholar.google.com/citations?user=MlJ_2-wAAAAJ&hl=en}{\textit{Assoc.Prof.PhD. Tung Lam Nguyen}}}
		{\begin{soloitemize}
				\item Field: Optimization, Robotics, and Control Systems.
%				\item Developed mathematical models and simulation environments for 3D ballbot systems, focusing on nonlinear dynamics, trajectory generation, and safety constraints.
				\item Developed mathematical models, simulation environments, advanced control and trajectory planning methods for ballbot navigation, including observer-based hierarchical sliding mode control, NMPC with control barrier functions (CBFs), and flatness-based time-optimal motion planning, with outcomes published in the International Journal of Robust and Nonlinear Control (RNC) and IEEE Access.
%				\item Designed and implemented advanced control algorithms, including observer-based hierarchical sliding mode control and nonlinear model predictive control (NMPC) with control barrier functions (CBFs) for obstacle avoidance and tilt angle limitation.
%				\item Formulated and solved time-optimal trajectory planning problems using flatness theory and optimization techniques, enabling smooth and efficient motion planning for ballbot navigation.
%				\item Authored and co-authored peer-reviewed journal papers based on the project outcomes, including publications in the International Journal of Robust and Nonlinear Control (RNC) and IEEE Access.
			\end{soloitemize}
		}

	
	\section*{Academic activities}
	\activities{Invited review for}{\textit{Nonlinear Dynamics (this is my first time as a reviewer)}}
	
	\sepspace
	
	\activities{Seminars and Talks}
	{
		\textit{2025: Talk "MuJoCo for Advanced Physics Simulation: From manipulators to autonomous vehicles" for "Motion Control" master course at HUST and MoCAR seminar} [\href{https://drive.google.com/file/d/10EOLlFqleqqPBXlAqDkhnFmAycjfnl9E/view?usp=drive_link}{pdf}] \\
		
		\noindent\textit{2025: Seminar "Underwater Vehicles" for modeling training of Autonomous Underwarter Vehicle at MEG-MoCAR} [\href{https://drive.google.com/file/d/13BD5C82OyaQ9N83s5FF_MnSdGozZ3Q1_/view?usp=drive_link}{pdf}]
	}


%	\section*{Conferences}
%	\conference{IEEE 12th International Conference on Control, Automation and Information Sciences (IEEE ICCAIS 2023)}{\textit{Hanoi, Vietnam}}
%	
%	\sepspace
%	
%	\conference{2024 International Conference on Advanced Technologies for Communications (IEEE ATC2024)}{\textit{Ho Chi Minh City, Vietnam}}
%	
%	\sepspace
%	
%	\conference{International Conference on Intelligent Systems and Networks (Springer ICISN 2023)}{\textit{Hanoi, Vietnam}}
	
	
	
	
	
	\section*{Honours \& awards}
	
	\conference{Master, PhD Scholarship Programme}{Vingroup Innovation Foundation (VINIF)}
	
	\sepspace
	
	\conference{Best Thesis Defense Award}{Hanoi University of Science and Technology}
	
	
	% ================= Skills Section =================
	\section*{Skills}
		\skill{Programming} {Python, C/C++, MATLAB for algorithms and embedded applications.}
		\skill{Simulation} {Simulink, Simscape, MuJoCo for modeling and dynamics.}
		\skill{Control \& Math} {Rigid body dynamics, motion control, optimization, GNC.}
		\skill{Engineering} {Git, PCB design, SolidWorks, experimental platforms.}
		\skill{Systems} {Real-time Linux, EtherCAT, embedded robotics/automation.}
		\skill{Research} {Publications, presentations, literature review, validation.}
%	\section*{Skills}
%	
%	\skill{Programming}{Proficient in Python, C/C++, and MATLAB for algorithm development, numerical computation, and embedded system applications.}
%	
%	\skill{Simulation}{Experienced with Simulink, Simscape, and MuJoCo for multi-domain physical modeling, robot dynamics simulation, and virtual prototyping.}
%	
%	\skill{Control \& Math}{Solid foundation in rigid body dynamics, control theories, motion control, optimization, and Guidance–Navigation–Control (GNC) systems.}
%	
%	\skill{Engineering}{Hands-on experience with version control (Git), PCB design and debugging, 3D CAD modeling using SolidWorks, and designing experimental platforms for validation.}
%	
%	\skill{Systems}{Familiar with EtherCAT-based Linux kernel development, real-time control architectures, and embedded systems programming for robotics and automation.}
%	
%	\skill{Research}{Capable of conducting scientific research, writing academic publications, and presenting technical findings at international conferences. Experienced in literature review, hypothesis formulation, and experimental validation.}
%	
	

	
\end{document}